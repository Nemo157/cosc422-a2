\section{Implementation}

  \subsection{Three Tone Shading}

    The three tone shading was simply implemented as a vertex and fragment
    shader pair.  The vertex shader transforms the normal vector by the normal
    matrix and records the light direction vector from the current vertex to the
    0th light source.  The fragment shader then takes the interpolated vectors,
    re-normalizes them and finds their dot product.  If the result of this is
    below 0.15 (determined experimentally) the colour used is 0.6 times the
    materials diffuse colour, if it's above 0.95 (again experimentally
    determined) then the colour used is 1.2 times the materials diffuse,
    otherwise the materials diffuse is used unchanged.

  \subsection{Pencil Shading}

    \subsubsection{Texture generation}

      The texture generation was simply implemented as stated.  The inbuilt
      random function was used to determine which rows to put the lines on.
      Each layer down the texture there were simply an extra 32 lines times the
      number of layers passed were added, so the top layer got no lines, the
      next got 32, the next got 64 + the original 32, the next 96 + the original
      96, etc.

    \subsubsection{Model Pre-processing}

      The pre-processing of the model was implemented mostly in the
      \texttt{Vert} class of the half-edge structure.  The lowest level
      operations were a part of the general geometry library, but the rest of
      the high level operations were left to be implemented directly in the
      \texttt{Vert\#calculate\_normals} method as they were not going to be
      re-used elsewhere.

      The \texttt{Face} class also contained a few parts of the pre-processing
      for converting the maximum primary curvature found in the Vert class to
      the required texture co-ordinates.


    \subsubsection{Run-time Processing}

      The run-time processing was implemented very similarly to the three tone
      shading.  A vertex shader computed the transformed normal vector and light
      direction at each vertex and passed the three texture co-ordinates through
      to the fragment shader.  These were interpolated by the OpenGL pipeline
      then the fragment shader calculated the dot product of the normal and
      light direction to get the third component of each texture co-ordinate.
      The texture was then accessed three times and the average of the colours
      at the point was used to colour the fragment.
